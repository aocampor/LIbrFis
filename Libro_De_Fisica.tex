\documentclass[10pt]{book}
\usepackage[spanish]{babel}
\usepackage[utf8]{inputenc}
\usepackage{framed}
\usepackage[usenames,dvipsnames,svgnames,table]{xcolor}
\usepackage{amsthm}
\usepackage{amsmath}
\usepackage{amsfonts}
\usepackage{amssymb}
\usepackage[framemethod=TikZ]{mdframed}
\usepackage{makeidx}
\usepackage{hyperref}  
\usepackage{tcolorbox}
\usepackage[tikz]{bclogo}
\usepackage{lipsum}

\newcommand{\me}{$m$}
\newcommand{\seg}{$s$}
\newcommand{\kg}{$Kg$}
\newcommand{\f}{$f$}
\newcommand{\inc}{$in$}
\newcommand{\gr}{$gr$}

\setlength{\topmargin}{-.5in}
\setlength{\textheight}{9in}
\setlength{\oddsidemargin}{.125in}
\setlength{\evensidemargin}{.625in}
\setlength{\textwidth}{6.025in}

\definecolor{bgblue}{RGB}{245,243,253}
\definecolor{ttblue}{RGB}{91,194,224}
\definecolor{mycolor}{rgb}{0.122, 0.435, 0.698}% Rule colour
\definecolor{bggreen}{rgb}{192,216,144}
\definecolor{bgred}{RGB}{255,150,153}

\theoremstyle{definition}

\newtheorem{xinn}{Pre-requisito}[chapter]
\newenvironment{prere}
  {\clubpenalty=10000
   \begin{xinn}%
   \mbox{}%
   %{\color{red}\leaders\hrule height .8ex depth \dimexpr-.8ex+0.8pt\relax\hfill}%
   %\mbox{}\linebreak\ignorespaces}
  }
  {%\par\kern2ex\hrule
   \end{xinn}
  }

\newtheorem{exinn}{Ejemplo}[chapter]

\newenvironment{example} 
 {\clubpenalty=10000 \begin{exinn} \mbox{}}
 {\end{exinn}}
  %  \mbox{}% 
  %  {\color{blue}\leaders\hrule height .8ex depth \dimexpr-.8ex+0.8pt\relax\hfill}%
  % \mbox{}\linebreak\ignorespaces}
  %{\par\kern2ex\hrule\end{exinn} }

\mdfdefinestyle{mystyle}{
rightline=true,innerleftmargin=10,innerrightmargin=10,
outerlinewidth=3pt,topline=false,rightline=true,bottomline=false,
skipabove=\topsep,skipbelow=\topsep
}

\makeatletter
\newcommand{\mybox}[1]{%
  \setbox0=\hbox{#1}%
  \setlength{\@tempdima}{\dimexpr\wd0+13pt}%
  \begin{tcolorbox}[colback=blue!5!white,colframe=blue!75!black,boxrule=0.5pt,arc=4pt,
      left=6pt,right=6pt,top=6pt,bottom=6pt,boxsep=0pt,width=\@tempdima]
    #1
  \end{tcolorbox}
}
\newcommand{\myboxr}[1]{%
  \setbox0=\hbox{#1}%
  \setlength{\@tempdima}{\dimexpr\wd0+13pt}%
  \begin{tcolorbox}[colback=red!5!white,colframe=red!75!black,boxrule=0.5pt,arc=4pt,
      left=6pt,right=6pt,top=6pt,bottom=6pt,boxsep=0pt,width=\@tempdima]
    #1
  \end{tcolorbox}
}
\newcommand{\myboxg}[1]{%
  \setbox0=\hbox{#1}%
  \setlength{\@tempdima}{\dimexpr\wd0+13pt}%
  \begin{tcolorbox}[colback=green!5!white,colframe=green!75!black,boxrule=0.5pt,arc=4pt,
      left=6pt,right=6pt,top=6pt,bottom=6pt,boxsep=0pt,width=\@tempdima]
    #1
  \end{tcolorbox}
}

\makeindex

\begin{document}
\title{F\'isica B\'asica: Mec\'anica y Din\'amica }
\author{A.Ocampo  \\
{\small\em \copyright \  Draft date \today }}
\date{ }
\maketitle
\addcontentsline{toc}{chapter}{Contents}
\pagenumbering{roman}
\tableofcontents
\listoffigures
\listoftables
\Large
\chapter{Introducci\'on}
Los humanos somos curiosos por naturaleza, preguntarnos como funciona nuestro entorno ha sido siempre tan com\'un, que para todo existe una explicaci\'on ya sea filos\'ofica, religiosa o cient\'ifica. A pesar de esto la ciencia como tal es un invento reciente fruto del esfuerzo, dedicaci\'on y sacrificio de muchas personas.\\
Los intentos por explicar el mundo y sus maravillas han sido hechos por hombres y mujeres desde el principio de la humanidad. Al inicio los mitos y las leyendes llenaban todos los vacios de conocimiento. Con el paso del tiempo y la sofisticaci\'on del pensamiento aparecieron las primeras corrientes filos\'oficas que empezaron a ofrecer explicaciones alternativas menos religiosas. La aparici\'on de las matema\'aticas tuvo un papel fundamental en los inicios de la comprension de la naturaleza, hasta el punto en el que sin ellas el lenguaje no ser\'ia suficiente para describir lo que pasa a nuestro alrededor.\\
El primer intento famoso por explicar el mundo es probablemente el tratado de aristoteles llamado \textit{F\'isica}. En este libro se empezaron a ofrecer explicaciones de las causas de algunos fenomenos, sin embargo, este tratado estaba a\'un lejos de lo que hoy en d\'ia conocemos como ciencia. Para llegar al estado actual se necesitaron varios cientos de a\~nos, un m\'etodo cient\'ifico riguroso y la alternaci\'on y validaci\'on mutua de hip\'otesis y experimentos para crear teor\'ias y modelos del mundo que nos rodea.\\
La palabra F\'isica viene del griego $\phi\upsilon\sigma\iota\kappa\alpha$ que significa \textit{natural}. Por lo tanto la f\'isica puede ser definida como el estudio de la naturaleza. Esta definici\'on es muy amplia y encierra todas las otras ciencias naturales, raz\'on por las cual se requiere ser un poco mas espec\'ifico al definirla. Hoy en d\'ia la F\'isica se define como la ciencia que estudia la mater\'ia, la energ\'ia y las leyes fundamentales que la gobiernan. Aunque esta definici\'on sigue siendo muy amplia, es mas concisa y deja espacio para el resto de las ciencias naturales tales como la biolog\'ia y la qu\'imica. En mi opini\'on personal la ciencia no tiene divisiones y en realidad las ramas modernas de la ciencia son mas generadas por la imposibilidad de estudiar todo el conocimiento que hemos acumulado y por los distintos intereses cient\'ificos de diferentes personas.\\
Con este libro espero recopilar todo el conocimiento que tengo al respecto de la f\'isica. Espero hacerlo de una manera did\'actica y comprensiva que lleve al lector de un nivel b\'asico principiante a un nivel avanzado con cierto grado de paciencia. Espero que disfruten la lectura de este libro tanto como yo disfruto la escritura del mismo.\\ 

\chapter{Bases}
intro base\cite{abramowitz+stegun}

\section{Sistemas de medida.}
\label{sec_MKS}
\begin{prere}
\label{pre_MKS}
\begin{tcolorbox}[colback=blue!5!white,colframe=blue!75!black,boxrule=0.5pt,arc=4pt, left=6pt,right=6pt,top=6pt,bottom=6pt,boxsep=0pt] 
  \textbf{Los pre-requisitos para este tema son:}\\
  Saber Leer.
\end{tcolorbox} 
\end{prere}
Todas las personas tienen puntos de vista distintos y describe los objetos que los rodean de manera subjetiva. Esto no es suficiente para describir el mundo de manera inambigua. Es por esto que se necesita una escala exacta que permita comparar objetos o propiedades entre si. Los humanos hemos creado escalas para medir cantidades pr\'acticamente desde que iniciamos nuestra existencia, y estas unidades han dependido de nuestra cultura y el tiempo en la historia en el que vivimos, por lo tanto entre dos cultures diferentes las escalas no coinciden. Para evitar este tipo de problemas se invent\'o el sistema de unidades internacional o sistema m\'etrico decimal tambi\'en llamado sistema MKS.\\
En este sistema se usa el kilogramo como medida de masa, el metro como medida de distancia, y el segundo como medida de tiempo.\\
\begin{table}[h]
\label{tab_MKS}
\huge
\begin{center}
\begin{tabular}{|lcr|}
\hline
 & & \\
Distancia & $\rightarrow$ & metros [\me] \\
Masa & $\rightarrow$ & Kilogramos [\kg] \\
Tiempo & $\rightarrow$ & Segundos [\seg] \\
 & & \\
\hline
\end{tabular}
\end{center}
\caption{Sistema M\'etrico Decimal tambien llamado sistema MKS o sistema internacional.} 
\end{table}
El sistema MKS no es el \'unico, existen otros sistemas usados con alta frecuencua como el sistema ingles el cual utiliza los pies o feet (\f)  para medida de distancia, gramos (\gr) para medida de masa y segundos (\seg) para medida de tiempo.\\ 
\begin{table}[h]
\huge
\begin{center}
\begin{tabular}{|lcr|}
\hline
 & & \\
Distancia & $\rightarrow$ & Feet [\f] \\
Masa & $\rightarrow$ & Gramos [\gr] \\
Tiempo & $\rightarrow$ & Segundos [\seg] \\
 & & \\
\hline
\end{tabular}
\label{tab_SI}
\end{center}
\caption{Sistema Ingles.} 
\end{table}
Hay que tener en cuenta que las cantidades listadas en las tablas \ref{tab_MKS} y \ref{tab_SI} no conforman un lista exhaustiva ya que hay mas cantidades de que no estan descritas ac\'a. A medida que vayamos avanzando en los temas se ir\'an listando mas cantidades relevantes, una tabla mas completa puede ser vista en el ap\'endice \ref{tab_app_MKS}.

\section{Ordenes de magnitud.}
\label{sec_Ordenes_Magnitud}
\begin{prere}
\label{pre_ord_mag}
\begin{tcolorbox}[colback=blue!5!white,colframe=blue!75!black,boxrule=0.5pt,arc=4pt, left=6pt,right=6pt,top=6pt,bottom=6pt,boxsep=0pt] 
  \textbf{Los pre-requisitos para este tema son:}\\
  Saber sumar, restar, multiplicar, dividir mas los pre-requisitos de la secci\'on \ref{pre_MKS}.
\end{tcolorbox} 
\end{prere}


\section{Conversi\'on de unidades.}
conversiones

\section{Problemas de estimaci\'on}
\begin{prere}
\begin{tcolorbox}[colback=blue!5!white,colframe=blue!75!black,boxrule=0.5pt,arc=4pt, left=6pt,right=6pt,top=6pt,bottom=6pt,boxsep=0pt] 
  \textbf{Los pre-requisitos para este tema son:}\\
  Ser curioso, mas los requisitos de la secci\'on \ref{sec_MKS}.
\end{tcolorbox} 
\end{prere}
Los problemas de Fermi son problemas de estimaci\'on. Son \'utiles para obtener una idea de cantidades que son aparentemente imposibles de calcular de otro modo. Para resolver un problema de fermi se necesita asumir el valor de una o mas cantidades de manera aproximada para luego realizar operaciones matem\'aticas simples que dan una respuesta muy cercana a la realidad en orden de magnitud. 
\begin{example}
\begin{tcolorbox}[colback=green!5!white,colframe=green!75!black,boxrule=0.5pt,arc=4pt,left=6pt,right=6pt,top=6pt,bottom=6pt,boxsep=0pt]
Algunos ejemplos son:\\
\begin{itemize}
\item Cuantos afinadores de piano hay en chicago?
\item Cuantas gotas de agua caben en un vaso?
\item Cuantas posibles parejas sentimentales tienes en la ciudad en la que habitas?
\item Cuantas veces ha latido tu coraz\'on?
\item Cuantos vasos de agua caben en una pisc\'ina?
\item Cuantas personas viven en tu vecindario?
\item Cuantos m\'edicos hay en tu ciudad?
\item Cuantas caras haz visto en tu vida?
\item etc....
\end{itemize}
\end{tcolorbox}

\end{example}
Resolvamos un par de problemas para entender como funciona.\\
\begin{example}
\begin{tcolorbox}[colback=green!5!white,colframe=green!75!black,boxrule=0.5pt,arc=4pt,left=6pt,right=6pt,top=6pt,bottom=6pt,boxsep=0pt]
\textbf{Problema:} \textit{Cuantas veces ha latido tu corazón?}\\
\textbf{Respuesta:}\\ 
Digamos que hoy es tu cumplea\~nos n\'umero $N$. Sabemos que cada a\~no tiene aproximadamente 365.25 d\'ias ($D_a$), por lo tanto has estado vivo durante $D$ d\'ias \\
\begin{equation}
D = N D_{a}.
\end{equation}
Asumamos que haces ejercicio $n$ veces por semana, y que cada sesi\'on de ejercício dura aproximadamente $t$ horas. Esto quiere decir que el numero de horas de ejercicio por semanas $n_e$ es:
\begin{equation}
n_e = nt.
\end{equation} 
Asumamos tambi\'en que durante el tiempo que haces ejercicio tu coraz\'on late $L_{e}$ veces por minuto, y durante el tiempo en el que no te ejercitas tu coraz\'on en reposo late aproximadamente $L_{r}$ veces por minuto. \\
Recordemos que cada semana tiene $n_d = 7$ d\'ias de $h_d = 24$ horas. Con estos datos sabemos que el numero de horas en una semana es:
\begin{equation}
n_s = n_d h_d = 168,
\end{equation}
Esto implica que el porcentaje de tiempo $p$ en el que haces ejercicio es aproximadamente las horas de ejercicio por semana sobre el n\'umero de horas en una semana:
\begin{equation}
p = \frac{n_e}{n_s}. 
\end{equation}
Por lo tanto haz hecho ejercicio durante $D_e$ d\'ias donde
\begin{equation}
D_e = D p, 
\end{equation} 
y haz estado en reposo durante
\begin{equation}
D_r = D - D_e, 
\end{equation} 
El n\'umero de latidos que tu coraz\'on ha dado aproximadamente es $N_L$:  
\begin{equation}
\label{eq_NLatidosDia}
N_L = D_e L_{e/dia}  + D_r L_{r/dia}, 
\end{equation} 
Como se tiene el numero de latidos por minuto y no por d\'ia, hay que aplicar un factor de conversi\'on. Reemplazando todos los datos que tenemos en la ecuaci\'on \ref{eq_NLatidosDia} obtenemos. 
\begin{align}
N_L & = D_e D_m L_e + ( D D_m - D_e D_m ) L_r\\
    & = D_m (Dp L_e + (D - Dp)L_r)\\
    & = D D_m ( L_r + p (L_e - L_r ) )
\end{align}
donde $D_m$ es el n\'umero de minutos por d\'ia. Supongamos que hoy es tu cumplea\~nos n\'umero 20, que haces ejercicio durante 2 horas 2 veces por semana, que durante el ejercicio tu coraz\'on late en promedia 140 veces por minuto, y que en reposo tu coraz\'on late aproximadamente 90 veces por minuto. Estos datos implican que llevas vivo 7305 d\'ias, haces 4 horas de ejercicio por semana y por lo tanto, el porcentaje de tiempo que haces ejercicio es de $2.3 \%$. Como un dia tiene 1440 minutos, se obtiene que el n\'umero de pulsaciones que has tenido en la vida aproximadamente es de:
\begin{align*}
N_L & = 7305 [dias] 1440 [\frac{min}{dias}] ( 90 [\frac{pulsos}{min}] + 0.023 (120 - 90)[\frac{pulsos}{min}] ) \\
    & = 10519200 [min] 90.69 [\frac{pulsos}{min}]  \\
    & = 953986248 [pulsos] 
\end{align*} 
Lo cual quiere decir que tu coraz\'on ha latido $953\times 10^6$ veces!. Esta estimaci\'on puede hacerse mas precisa, por ejemplo podr\'ias asumir que cuando estas dormido tu coraz\'on late 50 veces por minuto, etc.... sin embargo, esta estimaci\'on es aproximadamente correcta, ya que el orden de magnitud no va a cambiar mucho as\'i seas un deportista.
\end{tcolorbox}
\end{example}

\appendix
\chapter{Tablas completas de los sistemas MKS y Ingl\'es.}
\begin{table}[h]
\label{tab_app_MKS}
\huge
\begin{center}
\begin{tabular}{|lcr|}
\hline
 & & \\
Distancia & $\rightarrow$ & metros [\me] \\
Masa & $\rightarrow$ & Kilogramos [\kg] \\
Tiempo & $\rightarrow$ & Segundos [\seg] \\
 & & \\
\hline
\end{tabular}
\end{center}
\caption{Sistema M\'etrico Decimal tambien llamado sistema MKS o sistema internacional.} 
\end{table}

\begin{table}[h]
\label{tab_app_SI}
\huge
\begin{center}
\begin{tabular}{|lcr|}
\hline
 & & \\
Distancia & $\rightarrow$ & Feet [\f] \\
Masa & $\rightarrow$ & Gramos [\gr] \\
Tiempo & $\rightarrow$ & Segundos [\seg] \\
 & & \\
\hline
\end{tabular}
\end{center}
\caption{Sistema Ingles.} 
\end{table}

\bibliographystyle{amsplain}
\bibliography{bibliography}

\end{document}
