\label{sec_MKS}
\begin{prere}
\label{pre_MKS}
\begin{tcolorbox}[colback=blue!5!white,colframe=blue!75!black,boxrule=0.5pt,arc=4pt, left=6pt,right=6pt,top=6pt,bottom=6pt,boxsep=0pt] 
  \textbf{Los pre-requisitos para este tema son:}\\
  Saber Leer.
\end{tcolorbox} 
\end{prere}
Todas las personas tienen puntos de vista distintos y describe los objetos que los rodean de manera subjetiva. Esto no es suficiente para describir el mundo de manera inambigua. Es por esto que se necesita una escala exacta que permita comparar objetos o propiedades entre si. Los humanos hemos creado escalas para medir cantidades pr\'acticamente desde que iniciamos nuestra existencia, y estas unidades han dependido de nuestra cultura y el tiempo en la historia en el que vivimos, por lo tanto entre dos cultures diferentes las escalas no coinciden. Para evitar este tipo de problemas se invent\'o el sistema de unidades internacional o sistema m\'etrico decimal tambi\'en llamado sistema MKS.\\
En este sistema se usa el kilogramo como medida de masa, el metro como medida de distancia, y el segundo como medida de tiempo.\\
\begin{table}[h]
\label{tab_MKS}
\huge
\begin{center}
\begin{tabular}{|lcr|}
\hline
 & & \\
Distancia & $\rightarrow$ & metros [\me] \\
Masa & $\rightarrow$ & Kilogramos [\kg] \\
Tiempo & $\rightarrow$ & Segundos [\seg] \\
 & & \\
\hline
\end{tabular}
\end{center}
\caption{Sistema M\'etrico Decimal tambien llamado sistema MKS o sistema internacional.} 
\end{table}
El sistema MKS no es el \'unico, existen otros sistemas usados con alta frecuencua como el sistema ingles el cual utiliza los pies o feet (\f)  para medida de distancia, gramos (\gr) para medida de masa y segundos (\seg) para medida de tiempo.\\ 
\begin{table}[h]
\huge
\begin{center}
\begin{tabular}{|lcr|}
\hline
 & & \\
Distancia & $\rightarrow$ & Feet [\f] \\
Masa & $\rightarrow$ & Gramos [\gr] \\
Tiempo & $\rightarrow$ & Segundos [\seg] \\
 & & \\
\hline
\end{tabular}
\label{tab_SI}
\end{center}
\caption{Sistema Ingles.} 
\end{table}
Hay que tener en cuenta que las cantidades listadas en las tablas \ref{tab_MKS} y \ref{tab_SI} no conforman un lista exhaustiva ya que hay mas cantidades de que no estan descritas ac\'a. A medida que vayamos avanzando en los temas se ir\'an listando mas cantidades relevantes, una tabla mas completa puede ser vista en el ap\'endice \ref{tab_app_MKS}.
