\label{sec_Conversion_unidades}
\begin{prere}
\begin{tcolorbox}[colback=blue!5!white,colframe=blue!75!black,boxrule=0.5pt,arc=4pt, left=6pt,right=6pt,top=6pt,bottom=6pt,boxsep=0pt] 
  \textbf{Los pre-requisitos para este tema son:}\\
  Mirar los pre-requisitos \ref{pre_ord_mag}.
\end{tcolorbox} 
\label{pre_Conv_U}
\end{prere}
Generalmente para resolver ciertos problemas se necesita cambiar cantidades dadas en un sistema a otro sistema. Por ejemplo cuando hablamos con un anglo parlante y le preguntamos sobre el clima y las temperaturas, muy probablemente tendremos que hacer cambios de grados fahrenheit a grados cent\'igrados, o tendremos que convertir millas en kilometros. A este tipo de operaciones simples se les llama conversi\'on de unidades.\\
Cualquier cantidad puede ser convertida de un sistema al otro por ejemplo, una medida tomada en pies puede convertirse f\'acilmente en metros y visceversa. Para realizar este tipo de conversi\'on lo \'unico que se necesita es ser cuidadoso y organizado. Se puede no ser ninguna de las dos, en cuyo case se requiere algo de habilidad. Lo primero que se necesita para convertir de una unidad a otra es conocer una equivalencia, por ejemplo, sabemos que 1\m es igual a 100 \cm o que 1 \km es 1.6 millas aproximadamente, o que 1 \ms son 0.001 \s.\\
Estas equivalencias nos permiten hallar un factor de conversion entre dos cantidades.
\begin{example}
\begin{tcolorbox}[colback=green!5!white,colframe=green!75!black,boxrule=0.5pt,arc=4pt,left=6pt,right=6pt,top=6pt,bottom=6pt,boxsep=0pt]
\textbf{Problema:} \textit{La distancia entre Gante B\'elgica y Ginebra suiza es de aproximadamente 804 \km.¿Cuanto es esto en Millas? }\\
\textbf{Respuesta:}\\ 
Sabemos que 1.6 \km = 1 \mile, por lo tanto $\frac{1 milla}{1.6 Km} = 1$. Por lo tanto:\\
\begin{align}
804 Km & = 804 Km \cdot 1 \\
       & = 804 \cancel{Km} \frac{1 milla}{1.6 \cancel{Km}}\\
       & = 502.5 millas
\end{align}
\end{tcolorbox}
\end{example}  
El ejemplo anterior es el caso m\'as sencillo y m\'as usado. A diario las personas convierten dolares en euros, centigrados en fahrenheit, \km en \mile, galones en litros, etc \ldots\\
Hay que notar que no se puede converti un metro en un segundo o una temperatura en una distancia etc \ldots , a no ser que exista un factor de conversi\'on que relacione las dos cantidades. Por ejemplo en astrof\'isisca se dan distancias en a\~nos luz, esto solo tiene sentido si se tiene en cuenta que la velocidad de la luz es una constante y por lo tanto se pueden convertir tiempos en distancias multiplicando por la velocidad.\\
Para hacer conversionas mas compleajas solo basta con aplicar tantos factores de conversi\'on como sean necesarios, Por ejemplo:
\begin{example}
\begin{tcolorbox}[colback=green!5!white,colframe=green!75!black,boxrule=0.5pt,arc=4pt,left=6pt,right=6pt,top=6pt,bottom=6pt,boxsep=0pt]
\textbf{Problema:} \textit{Un carro viaja a una rapidez de 120 \km/\h, ¿A Cuanto equivale esto en Millas por hora?, y ¿en $cm$/\s?}\\
\textbf{Respuesta:}\\ 
Sabemos que 1.6 \km = 1 \mile, entonces\\
\begin{align}
120 \frac{Km}{h} & = 120 \frac{Km}{h} \cdot 1 \\
                 & = 120 \frac{\cancel{Km}}{h} \frac{1 milla}{1.6 \cancel{Km}} \\ 
                 & = 75 \frac{millas}{h}
\end{align}
\begin{align}
120 \frac{Km}{h} & = 120 \frac{Km}{h} \cdot 1 \cdot 1 \\
                 & = 120 \frac{\cancel{Km}}{\cancel{h}} \frac{1000 m}{1 \cancel{Km}}\frac{1 \cancel{h}}{3600 s} \\
                 & = \frac{120 \cdot 1000 m}{3600 s}
                 & = 33.3 \frac{m}{s}
\end{align}
\end{tcolorbox}
\end{example}  






