Los humanos somos curiosos por naturaleza, preguntarnos como funciona nuestro entorno ha sido siempre tan com\'un, que para todo existe una explicaci\'on ya sea filos\'ofica, religiosa o cient\'ifica. A pesar de esto la ciencia como tal es un invento reciente fruto del esfuerzo, dedicaci\'on y sacrificio de muchas personas.\\
Los intentos por explicar el mundo y sus maravillas han sido hechos por hombres y mujeres desde el principio de la humanidad. Al inicio los mitos y las leyendes llenaban todos los vacios de conocimiento. Con el paso del tiempo y la sofisticaci\'on del pensamiento aparecieron las primeras corrientes filos\'oficas que empezaron a ofrecer explicaciones alternativas menos religiosas. La aparici\'on de las matema\'aticas tuvo un papel fundamental en los inicios de la comprension de la naturaleza, hasta el punto en el que sin ellas el lenguaje no ser\'ia suficiente para describir lo que pasa a nuestro alrededor.\\
El primer intento famoso por explicar el mundo es probablemente el tratado de aristoteles llamado \textit{F\'isica}. En este libro se empezaron a ofrecer explicaciones de las causas de algunos fenomenos, sin embargo, este tratado estaba a\'un lejos de lo que hoy en d\'ia conocemos como ciencia. Para llegar al estado actual se necesitaron varios cientos de a\~nos, un m\'etodo cient\'ifico riguroso y la alternaci\'on y validaci\'on mutua de hip\'otesis y experimentos para crear teor\'ias y modelos del mundo que nos rodea.\\
La palabra F\'isica viene del griego $\phi\upsilon\sigma\iota\kappa\alpha$ que significa \textit{natural}. Por lo tanto la f\'isica puede ser definida como el estudio de la naturaleza. Esta definici\'on es muy amplia y encierra todas las otras ciencias naturales, raz\'on por las cual se requiere ser un poco mas espec\'ifico al definirla. Hoy en d\'ia la F\'isica se define como la ciencia que estudia la mater\'ia, la energ\'ia y las leyes fundamentales que la gobiernan. Aunque esta definici\'on sigue siendo muy amplia, es mas concisa y deja espacio para el resto de las ciencias naturales tales como la biolog\'ia y la qu\'imica. En mi opini\'on personal la ciencia no tiene divisiones y en realidad las ramas modernas de la ciencia son mas generadas por la imposibilidad de estudiar todo el conocimiento que hemos acumulado y por los distintos intereses cient\'ificos de diferentes personas.\\
Con este libro espero recopilar todo el conocimiento que tengo al respecto de la f\'isica. Espero hacerlo de una manera did\'actica y comprensiva que lleve al lector de un nivel b\'asico principiante a un nivel avanzado con cierto grado de paciencia. Espero que disfruten la lectura de este libro tanto como yo disfruto la escritura del mismo.\\ 
